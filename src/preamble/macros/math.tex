% Define a new math-specific command without overriding existing ones in text mode.
\DeclareDocumentCommand \NewMathCommand {>\TrimSpaces m >\TrimSpaces m m}
  {%
    \expandafter\let\csname old\string#1\endcsname=#1
    \expandafter\NewDocumentCommand\csname new\string#1\endcsname{#2}{#3}
    \DeclareRobustCommand #1
      {%
        \ifmmode
            \expandafter\let\expandafter\next\csname new\string#1\endcsname
        \else
            \expandafter\let\expandafter\next\csname old\string#1\endcsname
        \fi
        \next%
      }
  }

% Define a new math-specific abbreviation.
\NewDocumentCommand \Abbr {>\TrimSpaces m >\TrimSpaces m}
  {%
    \NewMathCommand #1 {} {#2}
  }

% Define a new named category.
\NewDocumentCommand \Categ {>\TrimSpaces m >\TrimSpaces m}
  {%
    \NewMathCommand #1 {>\TrimSpaces o}
      {\IfValueTF{##1}{\mathrm{#2}(##1)}{\mathbf{#2}}}
  }

\Abbr \Aut {\operatorname{Aut}}
\Abbr \ap {\mathsf{ap}}
\Abbr \bbC {\mathbb{C}}
\Abbr \bbi {\mathbb{i}}
\Abbr \bbI {\mathbb{I}}
\Abbr \bbZ {\mathbb{Z}}
\Abbr \bC {\mathbf{C}}
\Abbr \bD {\mathbf{D}}
\Abbr \bI {\mathbf{I}}
\Abbr \C {\mathscr{C}}
\Abbr \calU {\mathcal{U}}
\Abbr \coker {\operatorname{coker}}
\Abbr \colim {\operatorname{\underset{\longleftarrow}{lim}}}
\Abbr \dim {\operatorname{dim}}
\Abbr \D {\mathscr{D}}
\Abbr \End {\mathrm{End}}
\Abbr \empty {\varnothing}
\Abbr \eqq {\overset{\mathclap{\text{\tiny{def}}}}{=}}
\Abbr \Hom {\mathrm{Hom}}
\Abbr \id {\mathrm{id}}
\Abbr \I {\mathcal{I}}
\Abbr \im {\operatorname{im}}
\Abbr \ker {\operatorname{ker}}
\Abbr \M {\mathcal{M}}
\Abbr \N {\mathbb{N}}
\Abbr \one {\mathbf{1}}
\Abbr \op {{}^{\mathsf{op}}}
\Abbr \Q {\mathbb{Q}}
\Abbr \rank {\mathrm{rank}}
\Abbr \R {\mathbb{R}}
\Abbr \S {\mathbb{S}}
\Abbr \tensor {\otimes}
\Abbr \tr {\mathsf{tr}}
\Abbr \two {\mathbf{2}}
\Abbr \zero {\mathbf{0}}
\Categ \Ab {Ab}
\Categ \Cat {Cat}
\Categ \CMon {CMon}
\Categ \Grp {Grp}
\Categ \Lat {Lat}
\Categ \Mon {Mon}
\Categ \Pos {Pos}
\Categ \PSh {PSh}
\Categ \Set {Set}
\Categ \sSet {sSet}
\Categ \Top {Top}

\NewMathCommand \set {>\TrimSpaces m >\TrimSpaces o}
  {%
    \left\lbrace#1\IfValueT{#2}{ \mid #2}\right\rbrace%
  }

\NewMathCommand \Mod {s >\TrimSpaces m}
  {%
    \IfBooleanTF{#1}{\text{Mod-$#2$}}{\text{$#2$-Mod}}%
  }
