% -- amsthm.
\numberwithin{equation}{subsection}

\makeatletter
  \NewDocumentCommand \@NewThm {m m m m}
    {%
      \IfValueTF{#2}
        {\newtheorem{#3\IfBooleanT{#1}{*}}{#4}[#2]}
        {\newtheorem{#3\IfBooleanT{#1}{*}}[equation]{#4}}
      \newtheorem*{#3\IfBooleanF{#1}{*}}{#4}

      % Special enviroment for lists (i.e. itemize and enumerate).
      \NewDocumentEnvironment {#3-list} {o}
        {\IfValueTF{##1}{\begin{#3}[##1]}{\begin{#3}}\leavevmode\nopagebreak}
        {\end{#3}}
      \NewDocumentEnvironment {#3-list*} {o}
        {\IfValueTF{##1}{\begin{#3*}[##1]}{\begin{#3*}}\leavevmode\nopagebreak}
        {\end{#3*}}
    }

  % Create a whole class of new theorem environments.
  \NewDocumentCommand \NewThm {s o >\TrimSpaces m >\TrimSpaces m >\TrimSpaces o}
    {%
      \iflanguage{french}{
        \@NewThm{#1}{#2}{#3}{\IfValueTF{#5}{#5}{#4}}
        \@NewThm{#1}{#2}{#3s}{\IfValueTF{#5}{#5}{#4}s}
      }{
        \@NewThm{#1}{#2}{#3}{#4}
        \@NewThm{#1}{#2}{#3s}{#4s}
      }
    }
\makeatother

  \newtheoremstyle {runin}
    {0pt}
    {0pt}
    {}
    {}
    {}
    {~---~}
    {0em}
    {\thmnumber{(#2)}}

  \theoremstyle{runin}
  % A dirty hack to write numbered paragraphs.
  \newtheorem{numbered}[equation]{}


\newtheoremstyle {default}
  {}
  {}
  {}
  {}
  {}
  {.~---~}
  {0em}
  {\thmname{\textsc{#1}}\thmnumber{ #2}\thmnote{ (#3)}}

% Define \thfamily as \itshape if newtx is not used.
\ProvideDocumentCommand \thfamily {}
  {\itshape}

\newtheoremstyle {statement}
  {}
  {}
  {\thfamily}
  {}
  {}
  {.~---~}
  {0em}
  {\textbf{\thmname{#1}\thmnumber{ #2}}\thmnote{ (#3)}}

\newtheoremstyle {other}
  {}
  {}
  {}
  {}
  {}
  {.~---~}
  {0em}
  {\thmname{\emph{#1}}\thmnumber{ #2}\thmnote{ (#3)}}

\theoremstyle{default}
  \NewThm{conjecture}{Conjecture}
  \NewThm{definition}{Definition}[Définition]
  \NewThm{lemma}{Lemma}[Lemme]

\theoremstyle{statement}
  \NewThm{corollary}{Corollary}[Corollaire]
  \NewThm{theorem}{Theorem}[Théorème]
  \NewThm{proposition}{Proposition}

\theoremstyle{other}
  \NewThm{example}{Example}[Exemple]
  \NewThm{exercise}{Exercise}[Exercice]
  \NewThm{question}{Question}
  \NewThm{remark}{Remark}[Remarque]

% -- fancyhdr.
\renewcommand{\headrulewidth}{0pt}

% Default style.
\fancypagestyle {default}
  {%
    \fancyhf{}
    \fancyhead[LE,RO]{\small\thepage}
    \fancyhead[CE]{\small\leftmark}
    \fancyhead[RE]{\small§~\thesection}
    \fancyhead[CO]{\small\rightmark}
  }

\makeatletter
  \NewDocumentCommand \@TitleStyle {>\TrimSpaces m} {\microtypesetup{tracking=true}\textsc{\MakeUppercase{#1}}\microtypesetup{tracking=false}}

  \renewcommand{\sectionmark}[1]{\markboth{\@TitleStyle{\MakeUppercase{#1}}}{}}
  \renewcommand{\subsectionmark}[1]{\markright{\@TitleStyle{\MakeUppercase{#1}}}}

  % % Below is a dirty fix for the table of contents and the references.
  % \fancypagestyle {toc}
  %   {%
  %     \fancyhf{}
  %     \fancyhead[LE,RO]{\small\thepage}
  %     \fancyhead[CE]{\small\@TitleStyle{Contents}}
  %     \fancyhead[CO]{\small\rightmark}
  %   }

  % \fancypagestyle {bibliography}
  %   {%
  %     \fancyhf{}
  %     \fancyhead[LE,RO]{\small\thepage}
  %     \fancyhead[CE]{\small\@TitleStyle{References}}
  %     \fancyhead[CO]{\small\rightmark}
  %   }
\makeatother

% -- tikz-cd.
\tikzcdset{
  diagrams={
    column sep=2em,
    row sep=2em,
  }
}

\NewDocumentEnvironment {diagram} {>\TrimSpaces o}
  {\center\IfNoValueTF{#1}{\tikzcd}{\tikzcd[#1]}}
  {\endtikzcd\endcenter}

% -- titlecaps.
\iflanguage{french}{}{%
  \Addlcwords
    {%
      a amid an and anti as at atop but by down for from in into like near next nor of off on onto or out over pace past per plus qua save so than the till to up upon via with yet
    }
}

% -- titlesec.
\titleformat \section [block]
  {\center\Large}
  {§~\thesection.~~}
  {0pt}
  {\MakeUppercase}
  [\endcenter]

\titleformat \subsection
  {\large}
  {\thesubsection.~~}
  {0pt}
  {\emph}

% -- titletoc.
\titlecontents {section}
  [0em]
  {\vspace{1.5em}}
  {§~\thecontentslabel.~~\MakeUppercase}
  {\MakeUppercase}
  {\normalfont\titlerule*{~.~}\contentspage}

\titlecontents {subsection}
  [3em]
  {}
  {\thecontentslabel.~~\emph}
  {}
  {\normalfont\titlerule*{~.~}\contentspage}
