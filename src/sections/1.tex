\section{Chain Complexes}

\subsection{Complexes of $R$-Modules}

Homological algebra is a tool used in several branches of mathematics: algebraic topology, group theory, commutative ring theory, and algebraic geometry come to mind. It arose in the late 1800s in the following manner. Let $f$ and $g$ be matrices whose product is zero. If $g \cdot v = 0$ for some column vector $v$, say, of length $n$, we cannot always write $v = f \cdot u$. This failure is measured by the \emph{defect}
\[
  d = n - \operatorname{rank}(f) - \operatorname{rank}(g)
\]
In modern language, $f$ and $g$ represent linear maps
\begin{equation*}
  \begin{tikzcd}
    U \ar[r, "f"] & V \ar[r, "g"] & W
  \end{tikzcd}
\end{equation*}
with $gf = 0$, and $d$ is the dimension of the \emph{homology module}
\[
  H = \ker(g) / f(U)
\]

In the first part of this century, Poincaré and other algebraic topologists utilized these concepts in their attempts to describe \enquote{$n$-dimensional holes} in simplicial complexes. Gradually people noticed that \enquote{vector space} could be replaced by \enquote{$R$-module} for any ring $R$.

This being said, we fix an associative ring $R$ and begin again in the category $\mathbf{mod}\!\!-\!\!R$ of right $R$-modules. Given an $R$-module homomorphism $f : A \to B$, one is immediately led to study the kernel $\ker(f)$, cokernel $\operatorname{coker}(f)$, and image $\operatorname{im}(f)$ of $f$. Given another map $g : B \to C$, we can form the sequence
\begin{equation}
  \tag{$\ast$}
  \label{eq:seq}
  \begin{tikzcd}
    A \ar[r, "f"] & B \ar[r, "g"] & C
  \end{tikzcd}
\end{equation}
We say that such a sequence is \emph{exact} (at $B$) if $\ker(g) = \operatorname{im}(f)$. This implies in particular that the composite $gf : A \to C$ is zero, and finally brings our attention to sequences (\ref{eq:seq}) such that $gf = 0$.

\begin{definition}
  A \emph{chain complex} $C_\cdot$ of $R$-modules is a family $\{C_n\}_{n \in \mathbb{Z}}$ of $R$-modules, together with $R$-module maps $d = d_n : C_n \to C_{n - 1}$ such that each composite $d \circ d : C_n \to C_{n - 2}$ is zero. The maps $d_n$ are called the \emph{differentials} of $C_\cdot$. The kernel of $d_n$ is the module of \emph{$n$-cycles} of $C_\cdot$, denoted $Z_n = Z_n(C_\cdot)$. The image of $d_{n + 1} : C_{n + 1} \to C_n$ is the module of \emph{$n$-boundaries} of $C_\cdot$, denoted $B_n = B_n(C_\cdot)$. Because $d \circ d = 0$, we have
  \[
    0 \subseteq B_n \subseteq Z_n \subseteq C_n
  \]
  for all $n$. The \emph{$n$\textsuperscript{th} homology module} of $C_\cdot$ is the subquotient $H_n(C_\cdot) = Z_n / B_n$ of $C_n$. Because the dot in $C_\cdot$ is annoying, we will often write $C$ for $C_\cdot$.
\end{definition}

\begin{exercise}
  Set $C_n = \mathbb{Z} / 8$ for $n \geq 0$ and $C_n = 0$ for $n < 0$; for $n > 0$ let $d_n$ send $x\ (\operatorname{mod} 8)$ to $4x\ (\operatorname{mod} 8)$. Show that $C_\cdot$ is a chain complex of $\mathbb{Z} / 8$-modules and compute its homology modules.
\end{exercise}

There is a category $\mathbf{Ch}(\mathbf{mod}\!\!-\!\!R)$ of chain complexes of (right) $R$-modules. The objects are, of course, chain complexes. A \emph{morphism} $u : C_\cdot \to D_\cdot$ is a chain complex map, that is, a family of $R$-module homomorphisms $u_n : C_n \to D_n$ commuting with $d$ in the sense that $u_{n - 1} d_n = d_{n - 1} u_n$. That is, such that the following diagram commutes
\begin{equation*}
  \begin{tikzcd}
    \cdots \ar[r, "d"] & C_{n + 1} \ar[d, "u"] \ar[r, "d"] & C_n \ar[d, "u"] \ar[r, "d"] & C_{n - 1} \ar[d, "u"] \ar[r, "d"] & \cdots \\
    \cdots \ar[r, "d"] & D_{n + 1} \ar[r, "d"] & D_n \ar[r, "d"] & D_{n - 1} \ar[r, "d"] & \cdots
  \end{tikzcd}
\end{equation*}

\begin{exercise}
  Show that a morphism $u : C_\cdot \to D_\cdot$ of chain complexes sends boundaries to boundaries and cycles to cycles, hence maps $H_n(C\cdot) \to H_n(D\cdot)$. Prove that each $H_n$ is a functor from $\mathbf{Ch}(\mathbf{mod}\!\!-\!\!R)$ to $\mathbf{mod}\!\!-\!\!R$.
\end{exercise}

\begin{exercise}[Split exact sequences of vector spaces]
  Choose vector spaces $\{B_n, H_n\}_{n \in \mathbb{Z}}$ over a field, and set $C_n = B_n \oplus H_n \oplus B_{n - 1}$. Show that the projection-inclusions $C_n \to B_{n - 1} \subset C_{n - 1}$ make $\{C_n\}$ into a chain complex, and that every chain complex of vector spaces is isomorphic to a complex of this form.
\end{exercise}

\begin{exercise}
  Show that $\{\operatorname{Hom}_R(A, C_n)\}$ forms a chain complex of abelian groups for every $R$-module $A$ and every $R$-module chain complex $C$. Taking $A = Z_n$, show that if $H_n(\operatorname{Hom}_R(Z_n, C)) = 0$, then $H_n(C) = 0$. Is the converse true?
\end{exercise}

\begin{definition}
  A morphism $C_\cdot \to D_\cdot$ of chain complexes is called a \emph{quasi-isomorphism} (Bourbaki uses \emph{homologism}) if the maps $H_n(C_\cdot) \to H_n(D_\cdot)$ are all isomorphisms.
\end{definition}

\begin{exercise}
  Show that the following are equivalent for every $C_\cdot$:
  \begin{enumerate}
    \item $C_\cdot$ is \emph{exact}, that is, exact at every $C_n$.
    \item $C_\cdot$ is \emph{acyclic}, that is, $H_n(C_\cdot) = 0$ for all $n$.
    \item The map $0 \to C_\cdot$ is a quasi-isomorphism, where \enquote{$0$} is the complex of zero modules and zero maps.
  \end{enumerate}
\end{exercise}
